\documentclass[12pt,titlepage]{article}
\usepackage[utf8]{inputenc}
\usepackage[czech]{babel}


\begin{document}
\begin{titlepage}
\begin{center}
	\mbox{} \\[3cm]
	\Huge{Semestrální práce z předmětu KIV/UPS} \\[.5cm]
	\huge{Chatovací systém} \\[2.5cm]
	\Large{Tomáš Maršálek, A10B0632P} \\
	\large{marsalet@students.zcu.cz} \\[1cm]
	\normalsize{\today}
\end{center}
\thispagestyle{empty}
\end{titlepage}

\section{Zadání}
7. Chatovací systém. Realizujte programy serveru a klienta pro chatování.
Chatovaní server a klient bude podporovat přihlášení uživatele pod přezdívkou,
komunikaci s ostatními uživateli, pouze s jedním definovaným uživatelem a
odhlášení uživatele.  Protokol bude obsahovat příkazy LOGIN, LOGOUT, ALL\_MSG,
PRIV\_MSG, USERS, PING a odpovědi OK a ERR.

\section{Chatovací protokol}
\subsection{Přihlášení a odhlášení}
Po navázání spojení se serverem je nutné se přihlásit, aby si uživatele server přidal do seznamu.

Formát přihlašovací zprávy je:
\begin{verbatim}
LOGIN <jméno>
\end{verbatim}

Za uživatelské jméno bude použito první slovo které následuje za $LOGIN$. Není
možné používat jméno o více slovech. Pokud se přihlášení podařilo, tzn. jméno
již není používáno nebo nedošlo k jiným potížím, dostaneme odpověď $OK$ a od
této chvíle můžeme používat všechny funkce protokolu. V případě chybové
odpovědi $ERR$ můžeme opakovat přihlášení nebo ukončit spojení. K chatovací
funkcionalitě protokolu se ale nedostaneme přes správné přihlášení.

Pro odhlášení použijeme zprávu:
\begin{verbatim}
LOGOUT
\end{verbatim}
Pokud ukončíme spojení bez poslání této zprávy, server tento stav rozpozná a
uživatel je odstraněn ze seznamu uživatelů i bez korektního odhlášení.
Uživatelské jméno tak není zbytečně obsazeno.

\subsection{Formát vyměňovaných zpráv}
Uživatelé mohou komunikovat prostřednictvím veřejné nebo soukromé zprávy.
Veřejná zpráva je rozeslána všem uživatelům mimo odesílatele. Ve svém klientovi
tedy nedostane duplikátní zprávu.
\begin{verbatim}
ALL_MSG <zpráva>
\end{verbatim}
Veřejnou zprávu odešleme jednoduše pomocí tohoto dotazu. Zpráva může mít
maximální délku 2000 znaků, záleží na nastavení serveru.

Soukromou zprávu odešleme podobným příkazem, jen dodáme jméno příjemce.
\begin{verbatim}
PRIV_MSG <příjemce> <zpráva>
\end{verbatim}

Chceme-li si ověřit spojení se serverem, pošleme zprávu $PING$. Server jednoduše
odpoví zprávou $OK$.

Důležitou součástí protokolu je zjištění přihlášených uživatelů. Mohlo by se
zdát, že jde pouze o dodatečnou informaci, ale pro posílání soukromých zprávu
se jedná o nezbytnou funkcionalitu. Jednoduše pošleme požadavek na seznam uživatelů.

\begin{verbatim}
USERS
\end{verbatim}

Odpovědí je prostý seznam přihlášených uživatelů bez jakýchkoliv dodatečných
informací.
\begin{verbatim}
Pepa
Magda
Honza
Jirka
Luboš
Václav
Jiřina
\end{verbatim}

\section{Implementace} % TODO fakt?
\subsection{Server}
Server je C program využívající knihovnu Berkeley socketů a vláknovou knihovnu
pthread. 

Při inicializaci je vytvořen socket, který je připojen na výchozí nebo předem
zvolený port. Celý program je jeden nekonečný cyklus, který je obsluhován
funkcí $select$. Ta v každé iteraci cyklu vybere ten socket, ze kterého přišla
zpráva. V případě tohoto serveru existují dva typy socketů. Jedním z nich je
naslouchající socket, který přijímá nová spojení. Navázanému spojení je
přiřazen druhý typ socketu - uživatelský, ten odpovídá právě jednomu uživateli.
Server si uchovává seznam všech přihlášených uživatelů, jejich uživatelské
jméno a uživatelský socket, přes který s ním komunikuje. Pokud $select$ vybere
naslouchající socket, pouze vytvoří nový uživatelský socket a server čeká,
dokud z něj nepřijde požadavek o přihlášení. Po úspěšném přihlášení je přidán
do seznamu přihlášených uživatelů. V opačném případě, kdy $select$ vybere
uživatelský socket, server vyhodnotí požadavek a patřičně se zachová podle výše
uvedeného protokolu.

\subsection{Datové struktury}
Kromě uchovávání seznamu uživatelů je program relativně jednoduchý v ohledu
uchovávání dat. Data o uživately jsou uchována ve struktuře $user$, která nese
údaje o jeho přihlašovacím jménu a socketu, přes který s ním probíhá
komunikace. Díky funkci $getpeername()$ zjistíme IP adresu až když ji
potřebujeme, není třeba ji uchovávat, ale samozřejmě by to bylo možné.  Všichni
užavetelé jsou uchováni ve spojovém seznamu. Není očekáváno veliké množství
přihlášených uživatelů, proto složitější struktury by ani neměly příliš velký
dopad na efektivitu. V praxi se na vytížených chatovacích serverech setkáme s
počtem uživatelů v řádech stovek, pro vyšší počty už by bylo vhodné použít
stromové nebo hashovací struktury.


\section{Logování}
\subsection{Hlavní log}
Pro logování požadavků je vyhrazen speciální
soubor $server.log$, který loguje každý požadavek jako čas požadavku, IP adresu
žadatele a zprávu požadavku.

\subsection{Statistika}
Server si uchovává údaje o počtu přenesených bytů, počtu přenesených zpráv,
počtu navázaných spojení, počtu úspěšných i neúspěšných přihlášení, době běhu a
počet přenosů zrušených kvůli chybě. Tuto informaci zjistíme v interaktivním
módu. 

\section{Použité nástroje a prostředí}
Obě aplikace klienta i serveru byly vyvíjeny pod systémem GNU/Linux. Klient byl
vyvinut za pomoci IDE Eclipse, pro které byla prvotně vyvinuta GUI knihovna
SWT. Server a makefile byly napsány ve Vimu.
% TODO swt fakt?


\section{Uživatelká příručka}
Abychom vyzkoušeli tento chatovací protokol, musíme spustit server na určitém
portu a následně můžeme připojit libovolné množství klientů.

\subsection{Server}
\subsubsection{Přeložení}
Server je napsán v jazyce C pro Unixové prostředí. V adresáři se zdrojovými
soubory přeložíme buď pomocí build skriptu
\begin{verbatim}
$ make
\end{verbatim}

nebo ručně
\begin{verbatim}
$ gcc -pthread -o server *.c
\end{verbatim}

\subsubsection{Spuštění}
Server spustíme bez parametrů, bude použit výchozí port 1234. Port si můžeme zvolit spuštěním serveru s parametrem portu.

\begin{verbatim}
$ ./server
$ ./server 12345
\end{verbatim}

\subsubsection{Ovládání}
Po spuštění bude server běžet na pozadí interaktivního programu představenému
uživateli. Pomocí něj může nenásilně ukončit server, získat informace o
vyměněných datech nebo získat seznam právě přihlášených uživatelů.


\subsection{Klient}
Klient je program s grafickým uživatelským rozhraním v jazyce Java a využívá
GUI knihovnu SWT.
\subsubsection{Přeložení}
\subsubsection{Spuštění}
Protože se jedná o Java aplikaci, spuštění provedeme z konzole 
\begin{verbatim}
$ java -jar client.jar
\end{verbatim}

\subsubsection{Ovládání}
Při spuštění je uživateli představeno okno s jedinou možností - $Connection$,
kde vybere položku $Connect to$. Po zadání parametrů serveru (adresa, port a
uživatelské jméno) a potvrzení je při zadání korektních údajích a ještě
nepoužitého uživatelského jména úspěšně připojen a rovnou přihlášen, nemusí se
starat o záležitosti jako $LOGIN$ a $LOGOUT$.
\end{document}
